\documentclass[11pt]{article}
\usepackage[utf8]{inputenc}
\usepackage[T1]{fontenc}
\usepackage{fixltx2e}
\usepackage{graphicx}
\usepackage{longtable}
\usepackage{float}
\usepackage{wrapfig}
\usepackage{rotating}
\usepackage[normalem]{ulem}
\usepackage{amsmath}
\usepackage{textcomp}
\usepackage{marvosym}
\usepackage{wasysym}
\usepackage{amssymb}
\usepackage{hyperref}
\tolerance=1000
\usepackage{caption}
\usepackage{minted}
\usepackage[margin=0.75in]{geometry}
\usepackage{amssymb}
\usepackage{amsmath}
\usepackage{mathtools}
\usepackage{fancyvrb}
\usepackage{float}
\usepackage{enumerate}
\usepackage{multicol}
\usepackage{subcaption}
\usepackage{mdframed}
\usepackage{color} %red, green, blue, yellow, cyan, magenta, black, white
\definecolor{mygreen}{RGB}{28,172,0} % color values Red, Green, Blue
\definecolor{mylilas}{RGB}{170,55,241}
\usepackage[utf8]{inputenc}
\newcommand{\todo}{{\LARGE \emph{\color{red}TODO}}}
\author{Eric Crosson}
\date{Monday August 31, 2015}
\title{Probability PSet 1}
\hypersetup{
  pdfkeywords={},
  pdfsubject={Problem set regarding set computations}}
\newcounter{problem}
\newcounter{solution}

\newcommand\Problem{%
  \stepcounter{problem}%
  \textbf{\theproblem.}~%
  \setcounter{solution}{0}%
}

\newcommand\TheSolution{%
  \textbf{Solution:}\\%
}

\newcommand\ASolution{%
  \stepcounter{solution}%
  \textbf{Solution \thesolution:}\\%
}
\parindent 0in
\parskip 1em
\begin{document}
\maketitle
\surroundwithmdframed{minted}
\Problem{} We are given that $P(A) = 0.55, P(B^c) = 0.45$, and $P(A \cup{} B) =
  0.85$. Determine $P(B)$ and $P(A \cap{} B)$.
  
\Problem{} Let $A$ and $B$ be two sets.

\begin{enumerate}[(a)]
  \item Show that ${(A^c \cap{} B^c)}^c = A \cup{} B$ and ${(A^c \cup{} B^c)}^c = A \cap{} B$.
  \item Consider rolling a six-sided die once. Let $A$ be the set of outcomes
    where an odd number comes up. Let $B$ be the set of outcomes where a 1 or a
    2 comes up. Calculate the sets on both sides of the equalities in part (a),
    and verify that the equalities hold.
\end{enumerate}

\Problem{} Alice and Bob each choose at random a number between zero and two. We
  assume a uniform probability law under which the probability of an event is
  proportional to its area. Consider the following events:
  
\begin{enumerate}[A:]
  \item The magnitude of the difference of the two numbers is greater than $\frac{1}{3}$.
  \item At least one of the numbers is greater than $\frac{1}{3}$.
  \item The two numbers are equal.
  \item Alice's number is greater than $\frac{1}{3}$.
\end{enumerate}

Find the probabilities $P(A), P(B), P(A \cap{} B), P(C), P(D), P(A \cap{} D)$.

\Problem{} Show the formula $P \left( (A \cap{} B) \cup{} (A \cap{} B) \right) = P (A) + P (B) - 2P (A \cup{} B)$, which gives the probability that at least one of the events
  $A$ and $B$ will occur.
  
\Problem{} A person has forgotten the last digit of a telephone number, so he dials
  the number with the last digit randomly chosen. How many times does he have to
  dial (not counting repetitions) in order that the probability of dialing the
  correct number is more than $0.5?$
  
\Problem{} We roll two fair 6-sided dice. Each one of the 36 possible outcomes is
  assumed to be equally likely. 

\begin{enumerate}[(a)]
  \item Find the probability that doubles were rolled.
  \item Given that the roll resulted in a sum of 4 or less, find the conditional
    probability that doubles were rolled.
  \item Find the probability taht at least one die is a 6.
  \item Given that the two dice land on different numbers, find the conditional
    probability that at least one die is a 6.
\end{enumerate}

\end{document}
